\medskip
\noindent{\bf Zadanie 6.} 
\medskip

Pierwsza urna zawiera cztery białe i jedną czarną kulę, druga - 2 białe i 3 czarne. Losujemy urnę tak, by szansa wybrania pierwszej urny była dwukrotnie mniejsza niż drugiej. Następnie z wybranej urny losujemy kulę. Jakie jest prawdopodobieństwo wybrania kuli białej?\\\\
\textbf{Rozwiązanie:} \\\\
Prawdopodobieństwo wybrania pierwszej urny:
$$
P(A)=\frac{1}{3}
$$
Prawdopodobieństwo wybrania drugiej urny:
$$
P(B)=\frac{2}{3}
$$
Prawdopodobieństwo wylosowania białej kuli z pierwszej urny:
$$
P(C)= \frac{4}{5}
$$
Prawdopodobieństwo wylosowania białej kuli z drugiej urny:
$$
P(D)= \frac{2}{5}
$$
Prawdopodobieństwo wylosowania kuli białej:
$$
P(E)= P(A)\cdot P(C) + P(B) \cdot P(D)=\frac{1}{3} \cdot \frac{4}{5} + \frac{2}{3} \cdot \frac{2}{5}= \frac{4}{15} + \frac{4}{15} = \frac{8}{15}
$$
