\documentclass{article}
\usepackage{mathtools}
\usepackage{polski}
\usepackage[utf8]{inputenc}
\begin{document}


\[ \Omega =\left \{(o,o,o),(o,o,r),(o,r,o),(o,r,r),(r,o,o),(r,o,r),(r,r,o),(r,r,r) \right \} \] 
Prawdopodobieństwo wylosowania trzech reszek pod rząd:

\[ P\left(A\right)= \frac{1}{8} \] 
Prawdopodobieństwo wylosowania nieparzystej ilości reszek:
\[ P\left(B\right)= \frac{4}{8} \] 
Prawdoposobieństwo warunkowe zdarzenia A pod warunkiem zdarzenia B
\[P\left(A|B\right) = \frac{P\left(A \cap B\right)}{P\left( B\right)} = \frac{ P \left(\left \{(r,r,r) \right \} \cap \left \{(o,o,r),(o,r,o),(r,o,o),(r,r,r) \right \}\right)}{ P \left( \left \{(o,o,r),(o,r,o),(r,o,o),(r,r,r) \right \}\right)} = \] 
\[  \frac{\frac{1}{8}}{\frac{4}{8}} = \frac{1}{8} \cdot \frac{8}{4} = \frac{1}{4} \] 

 
\end{document}
