\subsection{Zadanie 3}

C - Chłopiec

D - Dziewczynka

$\Omega = \{(C,D), (C,C), (D,D), (D,C))\}$


\begin{itemize}

\item[a)] starsze dziecko jest chłopcem.
\\$A = \{(C,C)\}$ - rodzina z dwoma chłopcami.
\\B = \{(D,C), (C,C)\} - starsze dziecko jest chłopcem.
\\$A \cap B = \{(C,C)\}$
\\
\\$P(B) = \frac{1}{2}$
\\$P(A \cap B) = \frac{1}{4}$

$$P(A | B) = \frac{P(A \cap B)}{P(B)} = \frac{\frac{1}{4}}{\frac{1}{2}} = \frac{1}{2}$$
\item[b)] jest co najmniej jeden chłopiec.
\\A = \{(C,C)\} - rodzina z dwoma chłopcami.
\\B = \{(C,D), (C,C), (D,C)\} - co najmniej jeden chłopiec.
\\$A \cap B = \{(C,C)\}$
\\
\\$P(B) = \frac{3}{4}$
\\$P(A \cap B) = \frac{1}{4}$

$$P(A | B) = \frac{P(A \cap B)}{P(B)} = \frac{\frac{1}{4}}{\frac{3}{4}} = \frac{1}{3}$$

\end{itemize}

