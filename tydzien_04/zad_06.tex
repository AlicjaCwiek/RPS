\subsection{Zadanie 6}


\setlength\parindent{0pt}



Zadanie 6.  W klasie jest 20 osób przy czym dziewczyn jest o 6 więcej niż chłopców. Nauczyciel wybiera losowo do odpowiedzi cztery osoby, przy czym osoba raz wybrana nie jest pytania ponownie. Oblicz prawdopodobieństwo,że nauczyciel wybierze:\\



1. Samych chłopców
\\
2. Tyle samo dziewcząt co chłopców
\\


N - liczba osob w klasie, N = 20
\\
n - liczba osób pytanych, n = 4
\\
M - liczba chlopcow, M = 7
\\

Rozkład prawdopodobieństwa zmiennej losowej X o rozładzie hipergeometrycznym wyraża się wzorem:

\\

$$
P(X=k) = \dfrac{\binom{M}{k} * \binom{N-M}{n-k}}{\binom{N}{n}}
$$


1.\\
$
P(X=4)= \dfrac{\binom{7}{4} * \binom{13}{0}}{\binom{20}{4}}
$

2.
\\

Trzeba wybrać dwóch chłopaków z 7 i resztę osób z grupy dziewczyn
\\

\\
A - zdarzenie w którym nauczyciel wybral dwóch chłopców i dwie dziewczynki


$
P(A) = P(X=2)= \dfrac{\binom{7}{2} * \binom{13}{2}}{\binom{20}{4}}
$
