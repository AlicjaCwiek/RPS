\subsection{Zadanie 6}

W klasie jest 20 osób przy czym dziewczyn jest o 6 więcej niż chłopców.
Nauczyciel wybiera losowo do odpowiedzi cztery osoby, przy czym osoba raz
wybrana nie jest pytania ponownie. Oblicz prawdopodobieństwo, że nauczyciel
wybierze:

\begin{enumerate}
\item samych chłopców
\item tyle samo dziewcząt co chłopców
\end{enumerate}

$N$ - liczba osób w klasie, $N = 20$
$n$ - liczba osób pytanych, $n = 4$
$M$ - liczba chłopców, $M = 7$

Rozkład prawdopodobieństwa zmiennej losowej X o rozkładzie hipergeometrycznym
wyraża się wzorem:

$$P(X=k) = \dfrac{\binom{M}{k} * \binom{N-M}{n-k}}{\binom{N}{n}}$$

\begin{enumerate}
\item
$P(X=4)= \dfrac{\binom{7}{4} * \binom{13}{0}}{\binom{20}{4}}$

\item
Trzeba wybrać dwóch chłopaków z 7 i resztę osób z grupy dziewczyn

$A$ - zdarzenie w którym nauczyciel wybrał dwóch chłopców i dwie dziewczynki

$P(A) = P(X=2)= \dfrac{\binom{7}{2} * \binom{13}{2}}{\binom{20}{4}}$
\end{enumerate}