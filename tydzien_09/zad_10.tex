\documentclass{article}
\usepackage[utf8]{inputenc}
\usepackage{polski}
\begin{document}
\medskip 
\noindent{\bf Zad. 10}  
\begin{flushleft}
W - wolny
Z - zajety
$$
A= \left[ 
        \begin{array}{cc}
         0,8_{z/z} & 0,2_{z/w}\\ 
         0,1_{w/z} & 0,9_{w/w}
         \end{array}
      \right] 
      \qquad$$
      a)\\$$
      \left[ 
        \begin{array}{cc}
         0 & 1
         \end{array}
      \right] \cdot
      \left[ 
        \begin{array}{cc}
         0,8 & 0,2\\ 
         0,1 & 0,9
         \end{array}
      \right] 
      \cdot
      \left[ 
        \begin{array}{cc}
         0,8 & 0,2\\ 
         0,1 & 0,9
         \end{array}
      \right] = \left[ 
        \begin{array}{cc}
         0,17 & 0,83
         \end{array}
      \right] 
      \qquad 
      $$
      b)\\
      $$
      \left[ 
        \begin{array}{cc}
         \pi_1 & \pi_2
         \end{array}
      \right] \cdot
      \left[ 
        \begin{array}{cc}
         0,8 & 0,2\\ 
         0,1 & 0,9
         \end{array}
      \right]  = \left[ 
        \begin{array}{cc}
         \pi_1 & \pi_2
         \end{array}
      \right]$$
\begin{equation}
    \left\{\begin{array}{rcl}
                     0,8\cdot\pi_1+ 0,1\cdot\pi_2 = \pi_1\\
                     0,2\cdot\pi_1+ 0,9\cdot\pi_2 = \pi_2\\
                     \pi_1+ \pi_2 = 1
\end{array}\right.
\end{equation}
Po przeksztalceniach otrzymujemy: $\pi_1 = \frac{1}{3}, \pi_2 = \frac{2}{3}$


\end{flushleft}
\end{document}
