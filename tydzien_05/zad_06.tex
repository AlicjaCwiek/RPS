\medskip
\noindent{\bf Zad. 6} 

$$
F(x)
 = \left\{ \begin{array}{ll}
0.5 e^{ax} & \textrm{gdy $x \leq 1$}\\
bx+0.75 & \textrm{gdy $ 1 < x \leq 2 $}\\
1 & \textrm{gdy $2 < x$}
\end{array} \right.
$$

Ponieważ funkcja musi być ciągła, otrzymujemy następujące nierówności dla x w krańcach przedziałów:

$$    
 1) \:0.5 e^{ax}<bx+0.75,\: x=1 $$
$$0.5 e^{a}<b+0,75$$
$$ e^{a}<2b+1,5$$
$$a<ln(2b+1,5)$$

$$2) \:bx+0.75<1,\: x=2$$
$$2b+0,75<1$$
$$2b<\frac{1}{4}$$
$$b<\frac{1}{8}$$

Pochodna funkcji musi być dodatnia, więc:
$$3) \: (0.5 e^{ax})' = 0,5a e^{ax} \geq 0$$
$$a\geq 0$$
$$4) \: (bx+0.75)'=b\geq 0$$

Z nierówności 2 i 4 otrzymujemy przedział wartości b, wstawiając krańce tego przedziału do nierówności 1 otrzymamy przedział wartości a:
$$ 0 \leq b <\frac{1}{8}$$
$$ a<ln(\frac{3}{2}) \: \wedge \:  a<ln(\frac{7}{4}) \: \wedge \: a \geq 0 $$
$$ 0 \leq a <ln(\frac{3}{2})$$

a)$$
F(x)
 = \left\{ \begin{array}{ll}
0.5 e^{\frac{x}{2}} & \textrm{gdy $x \leq 1$}\\
0,1x+0.75 & \textrm{gdy $ 1 < x \leq 2 $}\\
1 & \textrm{gdy $2 < x$}
\end{array} \right.
$$

$$P(1 \leq x< 2) = F_{x} (2) - F_{x} (1) = \frac {2}{10} + \frac{3}{4} - \frac{1}{2} e^{\frac{1}{2}}=
\frac{19}{20}-\frac{1}{2} e^{\frac{1}{2}}$$

b)
$$P(0 \leq x\leq 1) = F_{x} (1) - F_{x} (0) = \frac{1}{2} e^{\frac{1}{2}}-\frac{1}{2} $$

c)
$$P(0,5 \leq x \leq 1,5) = F_{x} (\frac{3}{2}) - F_{x} (\frac{1}{2}) = \frac {3}{20} + \frac{3}{4} - \frac{1}{2} e^{\frac{1}{2}}=
\frac{9}{10}-\frac{1}{2} e^{\frac{1}{2}}$$

d)
$$P(-1 \leq x< 3) = F_{x} (3) - F_{x} (-1) = 1-\frac{1}{2} e^{\frac{1}{2}} $$


