\begin{zad}\newline
Mamy daną funkcje z niewiadomymi a, b, c:
	$F(x) = \left\{ \begin{array}{rcl} 0} & \mbox{dla} & x<1 \\ b(1-\frac{c}{x}) & \mbox{for} & 1< x \leq a \\ 1 & \mbox{for} & x \geq a \end{array}\right.} $ \newline

 Wiemy że w punkcie 1 równanie drugie przyjmuje wartosć 0. Podstawiamy pod x = 1 i otrzymujemy  0 = b(1 - c). Z drugiej strony zas gdy podstawimy górną granicę do drugiego równania (czyli x = a) dostajemy b(1 - $\frac{c}{a}$)
Z pierwszego rownania otrzymujemy: 0 = b(1-c) \Rightarrow b = 0 $\vee$ c = 1. Jako, że przy b = 0 dystrybuanta będzie nie ciągła zatem przyjmujemy że c = 1. 
Po podstawieniu c = 1 otrzymujemy $b(1-\frac{1}{a}$) = 1 i dzielimy obustronnie przez b. Dostajemy $1 - \frac{1}{a} = \frac{1}{b}$. Czyli a > 1. 
\newline
\begin{center} 
\b Odpowiedz: \newline
Zatem c = 1, a i b zas spełniają warunek $1 = \frac{1}{a} + \frac{1}{b}$ 
\end{enter}
\end{zad}
