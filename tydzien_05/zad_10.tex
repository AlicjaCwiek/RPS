\subsection{Zadanie 10}

Najpierw obliczymy z jakiego przedziału interesują nas argumenty. Zatem

\begin{center} 
	$|X - \frac{3}{2}| < 2 \iff 
	x-\frac{3}{2} < 2 \land x-\frac{3}{2} > -2 \iff  x < \frac{7}{2} \land x> -\frac{1}{2} $	
\end{center}

Nasza zmienna losowa jest określona na zbiorze [-1,0] $\cup$ [2, 4].
Dlatego przedział nasz zawęża się do  $[-\frac{1}{2}, 0] \cup [2, \frac{7}{2}]$.
Wiemy, iż całkowite pole pod wykresem ma wynosić 1.
Suma boków dwóch prostokątów wynosi 3. Zatem wysokosć obydwu figur będzie $\frac{1}{3}$.
Nasza gęstość więc wynosi $ f(x) = \frac{1}{3}$.
Teraz już możemy zająć się liczeniem prawdopodobieństwa.
W tym celu liczymy całkę:

\begin{center}
	$$ P(|X - \frac{3}{2}| < 2) = P(x \in (-\frac{1}{2},\frac{7}{2}) = \int_{\frac{-1}{2}}^{0} \frac{1}{3} dx  + \int_{2}^{\frac{7}{2}}\frac{1}{3} dx = ... = \frac{1}{3}(\frac{1}{2} + \frac{3}{2}) = \frac{2}{3}$$
\end{center}

Prawdopodobieństwo wynosi $\frac{2}{3}$.
