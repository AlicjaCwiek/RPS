\subsection{Zadanie 14}

\begin{flushleft}
P(X=k) - losowo wybrany moduł zawiera k błędów \\
Prawdopodobieństwo zmiennej losowej o rozkładzie Poissona z parametrem $\lambda$:
\begin{equation}
    P(X=k)=\left\{\begin{array}{rcl}
                     \frac{e^{-\lambda\cdot x}}{k!}&dla&k>0\\
                     0 & dla & k \le 0
\end{array}\right.
\end{equation}
Wyznaczam wartość prawdopodobieństwa przez zdarzenie przeciwne:\\
\begin{equation}
    P(X \ge 3)=1-P(X<3)=1-(P(X=0)+P(X=1)+P(X=2))
\end{equation}
\begin{center}
$\lambda$ = ?\\
\end{center}
\begin{flushleft}
Wyznacznam $\lambda$ wykorzystując fakt, że:\\
\end{flushleft}
\begin{equation}
\left\{\begin{array}{rcl}
EX=\sum_{i=1}^{n} x_i\cdot p_i\\
EX = \lambda
\end{array}\right.
\end{equation}
Czyli rozwiązując powyższy układ (3) wyznaczam $\lambda$:
\begin{equation}
\lambda = 0\cdot \frac{4}{20}+1\cdot \frac{3}{20}+2\cdot \frac{5}{20}+3\cdot \frac{2}{20}+4\cdot \frac{4}{20}+5\cdot \frac{1}{20}+6\cdot \frac{1}{20} = \frac{23}{10}
\end{equation}
\\
Więc podstawiając do (1) wartość $\lambda$ wyznaczoną w (4) otrzymujemy wzór na $P(X=k)$, który podstawiając do (2) da nam ostateczny wynik:\\ 
\begin{center}
$P(X\ge3)=0.4039611741$
\end{center}
\end{flushleft}
