\medskip
\noindent{\bf Zad. 8} 

$$
P(X > t+s | X > t ) = P(X > s)
$$


Dla rozkładu wykładniczego:
$$
P(X > t+s | X > t ) = \frac{P(X > s + t, X > t)}{P(X > t)} = \frac{P(X > s + t)}{P(X > t)} = \frac{e^{-\lambda(s + t)}}{e^{-\lambda t}} = e^{-\lambda s} = P(X > s)
$$

Dla rozkładu geometrycznego:
$$
P(X > t+s | X > t ) = \frac{P(X > s + t, X > t)}{P(X > t)} = \frac{P(X > s + t)}{P(X > t)} = \frac{(1-p)^{s + t}}{(1-p)^{t}} = (1-p)^{s} = P(X > s)
$$
